\documentclass[a4paper, 11pt]{article}

\usepackage[utf8x]{inputenc}
\usepackage[T1]{fontenc}
\usepackage{ucs}
\usepackage[english]{babel}
\usepackage{mathtools}
\usepackage{amsmath}
\usepackage{amsfonts}
\usepackage{ulem}
\usepackage{verbatim}
\usepackage{fancyhdr}
\usepackage[parfill]{parskip}
\usepackage{graphicx}
\usepackage{palatino}
\usepackage{float}
\usepackage[font={small,it}]{caption}

\linespread{1.05}
\pagestyle{fancyplain}
\fancyhead{}
\fancyfoot[L]{}
\fancyfoot[C]{}
\fancyfoot[R]{\thepage}
\renewcommand{\headrulewidth}{0pt}
\renewcommand{\footrulewidth}{0pt}
\setlength{\headheight}{13.6pt}

\widowpenalty=1000
\clubpenalty=1000

\newcommand{\horrule}[1]{\rule{\linewidth}{#1}}

\title{ 
\normalfont \normalsize 
\textsc{University of Copenhagen} \\ [25pt]
\horrule{0.5pt} \\[0.4cm]
\huge PCSD: Assignment 4 \\
\horrule{2pt} \\[0.5cm]
}

\author{Jens Fredskov (chw752)\\Henrik Bendt (gwk553)\\Ronni Lindsgaard (mxb392)} % Your name

\begin{document}
\maketitle
\pagebreak

\section{Communication Abstractions} % (fold)
\label{sec:communication_abstractions}

To utilize RPC for an asynchronous, persistent communication protocol we would use a server (similar to the email protocol) to collect and forward all communication between all procedures. This is done in the following way.

On the client side the API contains the methods \texttt{push} and \texttt{pull}. When a client comes online it immediately performs a pull-request to the server with its unique ID (could be a username, MAC address, etc.). If no messages are present the server returns ok, otherwise it returns all outstanding messages. When a client needs to communicate either a request or a response to another client it does so by sending it to the server with the \texttt{push} method.

On the server side the API contains the method \texttt{push}. When a client sends a message the server immediately returns ok (to avoid blocking). It then tries to send the message, using \texttt{push} to the receiver. If the receiver is not present, the message is stored on the server until the receiver it self issues a \texttt{pull}.

%TODO: limitations! important stuff

% section communication_abstractions (end)

\section{Reliability} % (fold)
\label{sec:reliability}

\begin{enumerate}
    \item The probability of the daisy chain connecting all buildings is the same as the probability of all links working which is $(1-p)^N$ where $N$ is the number of links.
    \item If we have $N+1$ buildings, we must have $\sum_{k=0}^N k = N(N-1) \over 2$ links. 
\end{enumerate}

% section reliability (end)

\section{Implementation} % (fold)
\label{sec:implementation}

\paragraph{NOTES} % (fold)
\label{par:notes}

The load balancing is done independently on both proxies, meaning that they might end up counteracting each other.

% paragraph notes (end)

% section implementation (end)

\end{document}